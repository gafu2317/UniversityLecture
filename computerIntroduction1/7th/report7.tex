\documentclass{jsarticle}
\usepackage[dvips]{graphicx}
\title{コンピューター入門 第7回課題}
\author{福富隆大 学籍番号35714121}
\date{2023年11月9日}
\parindent = 0pt
\begin{document}
\maketitle 
\section{課題4番目}
(ア)球をsphereという関数で描画した。球か円かを判別するためにnoStroke()を用いた。\\
(イ)rotateYだと横に回転したが、rotateXだと縦に回転した。\\
\section{課題5番目}
(ア)text関数のx座標を100にするとテキストが左に移動した\\
textsize関数を100にするとテキストが大きくなった\\
(イ)text関数の上にfill(255,0,0)を追加し文字を赤くした
\section{課題6番目}
(ア)初期条件を10に変えると、上側に円が増えた\\
(イ)終了条件を800に変えると、下側に円が増えた\\
\section{課題8番目}
int y = 0;\\
int dy = 5;\\
void setup(){\\
size(400, 400);\\
background(0);\\
stroke(255);\\
}\\
void draw(){\\
background(0);\\
ellipse(200,y,50,50);\\
y = y + dy;\\
if (y == 400) {\\
dy = 0;\\
}\\
}
\end{document}
