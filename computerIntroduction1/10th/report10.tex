\documentclass{jsarticle}
\usepackage[dvips]{graphicx}
\title{コンピューター入門 第10回課題}
\author{福富隆大 学籍番号35714121}
\date{2023年12月4日}
\parindent = 0pt
\begin{document}
\maketitle
\section{プログラムの説明}
スタート画面でゲームスタートの文字をクリックするとゲームが始まる。\\
wキーでジャンプ、sキーで落下速度アップ、aキーで左に移動、dキーで右に移動する。\\
左右はずっと移動し、氷の上を滑っているみたいになっている。\\
右から玉が出てくるのでそれを避ける。\\
玉に当たるとゲームオーバーになる。\\
50秒間生き残るとクリアになる。\\
ゲームオーバー、ゲームクリアになったら、スタート画面に戻れる。\\
\section{苦労した点}
日本語を使おうとするとエラーになった。\\
→フォントをしていすると解決した。\\
プレイヤー(正方形)と障害物(円)の当たり判定をどうするかが難しかった。\\
→正方形とその外の領域を9つに区切って考え、円の中心との距離を用いると上手く場合分けできた。\\
残り時間を表示させるときに、フレームレートを60に設定して1/60を残り時間から引き続けたがうまくいかなかった。 \\
→int型で定義していたので1/60 = 0となってしまい、時間が減らなかった。\\
何回もプレイできるように、ゲームオーバーになったらスタート画面に戻るようにしたが、プレイヤーの位置や時間が引き継がれていて続きからになってしまった。 \\
→位置や時間の変数を元の値に戻して解決した。\\
\section{問題点とその改良}
障害物がランダムなタイミングで出てくるので、運が悪いと何も出てこないことがある。
\\
障害物の速さや動きがランダムなので、運が悪いと避けられないことがある。\\
障害物の見た目がただの円なので、もっと見た目を工夫したい。\\
\end{document}
%pdfにするときはpltex->dvipdfmxでコンパイルする