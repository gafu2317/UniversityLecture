\documentclass{jsarticle}
\usepackage[dvips]{graphicx}
\title{コンピューター入門 第6回課題}
\author{福富隆大 学籍番号35714121}
\date{2023年11月6日}
\parindent = 0pt
\begin{document}
\maketitle
\section{課題2番目}
マウスを追従するように二つの円と線が描画されている、それらは重力の影響を受けている。
\section{課題3番目}
円が同じ幅と高さの四角形になった。
\section{課題12番目}
size(480, 120);\\
background(0);\\
noStroke();\\
ellipse(132, 82, 200, 200);\\
stroke(255);\\
noFill();\\
ellipse(228, -16, 200, 200);\\
fill(255);\\
\section{課題17番目}
演算子を+から/に変更した。\\
その結果+の時に隠れていた円が現れて、円が二つ描画された。
\section{課題18番目}
yの値が1ずつ増えていくので線のy座標が大きくなり、下に移動していく。
\section{課題19番目}
A:y座標を300にすると、立方体が下に移動した。\\
B:辺の長さを250に変更すると、立方体が大きくなった。\\
C:Y軸を60度回転させると、立方体が上から見て時計回りに60度回転した。\\
\section{課題20番目}
rotetoXとrotetoYを入れ替えてみると、横回転だったものが縦回転になった。
\section{課題21番目}
noStroke();をコメントアウトしたが肉眼では違いが分からなかった。背景と同じ色だったからだと思われる。
\section{課題22番目}
ambientLightの値を(20, 20, 20);から(200, 20, 20)に変更したら右の青い球が赤くなった。\\
lightSpecularの値を(255, 255, 255)から(0, 255, 255)に変更したら左の赤い球が黒くなった。\\
directionalLightの値を(100, 100, 100, 0, 1, -1)から(200, 100, 100, 0, 1, -1)に変更したら右の青い球が紫色っぽくなった。\\
specular(255, 255, 0);に変更すると、左の赤い球が黄色くなった。\\
\section{課題23番目}
14行目のshininessの値を(5.0)から(100.0)に変更すると右の球の光沢は変わらなかったが、左の球の光沢がほとんどなくなった。\\
20行目のshininessの値を(1.0)から(100.0)に変更すると、左の球に光沢は変わらなかったが、左の球の光沢がほとんどなくなった。\\
\end{document}