\documentclass{jsarticle}
\usepackage[dvips]{graphicx}
\title{コンピューター入門 第9回課題}
\author{福富隆大 学籍番号35714121}
\date{2023年11月日}
\parindent = 0pt
\begin{document}
\maketitle
\section{プログラムの説明}
lesson8.pdeが改善前のプログラムで、pongame.pdeが改善後のプログラムである。\\
板を動かしてボールを跳ね返すポンゲームを作成した。\\
左の板をキーボード操作、右の板をマウス操作で動かすことができる。\\
キーボードは板をwで上に、sで下に動かすことができる。\\
マウスはカーソルの位置で板の位置が決まる。\\
画面の中央にボールを出現させ、ボールの位置に合わせて板を動かし、ボールと接触させることでボールが跳ね返る。\\
板や上下左右との接触はif文を用いて判定している。\\
ボールを跳ね返すことができなかったら負けとなり、左のプレイヤーが負けたら画面が一瞬青くなり、右のプレイヤーが負けたら画面が一瞬赤くなる。\\
\section{不満点と改善方法}
\begin{itemize}
  \item ボールが画面中央から出現するときに、ボールが動き出す方向が一定なので、random関数を使ってランダムな方向に動き出すようにした。ただしx方向、y方向の動く速さが0や0に近い値だと困るのでwhile文を使って0に近い数値は除外した。\\
  \item ボールの速さが一定なのはつまらないので、跳ね返るときと中央に戻った時に速さが変わるようにした。\\
  \item ボールの当たり判定は中心の点のみだが、見た目からはわからにくくすり抜けたように見えてしまうことがあるので、ボールの中心を赤くし、まわりを半透明の白色にした。\\
  \item 跳ね返りを座標がちょうどの時に判定すると、すり抜けることがあるのである程度座標の幅を持たせて判定させた。
  \item キーボードでの移動が遅かったので、キーボードでの移動の速さをマウスでの移動の速さに合わせた。\\
  \item どちらが勝っているのか分かりにくかったので、スコアを表示させた。\\
\end{itemize}
\section{自己評価}
同じ処理を何回も書かなくていいようにvoid ~ (){}で一度書いてしまい、それを呼び出すようること分かりやすいコードにすることができた。\\
コメントをこまめに書くことで、後から見ても分かりやすいコードにすることができた。\\
ゲーム難易度などの調整はあまりできなかったが、すり抜けを防止したり、xやy方向の速度が0に近い場合を除外するなど、ゲームの仕様のせいでプレイヤーが困らないように努力した。\\
総合して考えると、面白いゲームではないかもしれないが、自分が考えた通りに動いてくれるゲームになるように工夫できたので、自己評価は高い。\\
改善点は自分以外の人にもプレイしてもらって自分だけでは気付けないバグの発見や、難易度調整もやっていきたい。\\
\end{document}