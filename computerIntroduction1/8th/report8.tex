\documentclass{jsarticle}
\usepackage[dvips]{graphicx}
\title{コンピューター入門 第8回課題}
\author{福富隆大 学籍番号35714121}
\date{2023年11月9日}
\parindent = 0pt
\begin{document}
\maketitle
\section{課題4番目}
(ア)マウスボタンを押しているときに色が変化するためにfill(255,0,0,)を変数R,G,B,を宣言したうえでfill(R,G,B)としました\\
if文の中にランダムに決定する関数を入れてしまうと、マウスボタンを押している間ずっと色が変化してしまうため、マウスボタンが押されたときに色が変化するようにするために、\\
void draw(){}の下に以下のコードを追加しました。\\
void mousePressed(){\\
  R = random(255);\\
  G = random(255);\\
  B = random(255);\\
}\\
そうするとマウスボタンが押されている間だけランダムな色に変化させることができました。
(イ)大きさを変える時も同じように変数を宣言して、マウスボタンが押されたときに変数を二倍にして大きさを変化させるようにしました。\\
\section{課題6番目}
(ア)Sキーが押されたとき落ちずに、Sキーが押されていないとき落ちない処理を作るために、まずブール値を宣言し、それを使いif文をSキーが押されている時に変えました。\\
そしてSキーが押された判定をするために以下のコードを追加しました。\\
void keyPressed() {\\
  if (key == 's' || key == 'S') {\\
    isSKeyPressed = true;\\
  }\\
}\\
void keyReleased() {\\
  if (key == 's' || key == 'S') {\\
    isSKeyPressed = false;\\
  }\\
}\\
その結果Sキーが押されている時だけ落ちないようにすることができました。\\
(イ)Sキーが押されている間だけ落ちないようにするために、以下のコードを追加しました。\\
(イ)
(ア)と同様にしてwキーが押された時の処理を書き、wキーが押されている間だけyが増えるようにして上に移動するようにしました。\\
\section{課題7番目}
プログラミングが好きなのでとても楽しく課題をやることができています。\\
もう少し難しい課題もやってみたいです。\\
\end{document}