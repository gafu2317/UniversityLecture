\documentclass{jarticle}

%%%%%%%%%%%%%%%%%%%%%%%%%%%%%%%%%%%%%%%%%%%%%%%%%%%%%%%%%%%%%%%%%%%%%%%%%%%
% packages
%%%%%%%%%%%%%%%%%%%%%%%%%%%%%%%%%%%%%%%%%%%%%%%%%%%%%%%%%%%%%%%%%%%%%%%%%%%
\usepackage{amsmath}
\usepackage{amssymb}
\usepackage{amsfonts}
\usepackage[dvipdfmx]{graphicx}
\usepackage[dvipdfmx]{color}
\usepackage{graphicx}
\usepackage{bm}

%%%%%%%%%%%%%%%%%%%%%%%%%%%%%%%%%%%%%%%%%%%%%%%%%%%%%%%%%%%%%%%%%%%%%%%%%%%
% format stuffs
%%%%%%%%%%%%%%%%%%%%%%%%%%%%%%%%%%%%%%%%%%%%%%%%%%%%%%%%%%%%%%%%%%%%%%%%%%%
\setlength{\oddsidemargin}{0.455cm} 
\setlength{\evensidemargin}{0.455cm} 
\setlength{\textwidth}{15.5cm} 
\setlength{\textheight}{22.54cm}
\setlength{\headheight}{0mm}
\setlength{\headsep}{0mm}
\setlength{\topskip}{0mm}
\setcounter{topnumber}{100}
\setcounter{bottomnumber}{100}
\setcounter{totalnumber}{100}
\renewcommand{\topfraction}{1.0}
\renewcommand{\bottomfraction}{1.0}
\renewcommand{\textfraction}{0.0}
\renewcommand{\floatpagefraction}{0.0}
\renewcommand{\baselinestretch}{1.0}
\pagestyle{empty}

%%%%%%%%%%%%%%%%%%%%%%%%%%%%%%%%%%%%%%%%%%%%%%%%%%%%%%%%%%%%%%%%%%%%%%%%%%%
% math symbols and commands
%%%%%%%%%%%%%%%%%%%%%%%%%%%%%%%%%%%%%%%%%%%%%%%%%%%%%%%%%%%%%%%%%%%%%%%%%%%
\newcommand{\eq}[1]{(\ref{#1})}
\newcommand{\mtx}[2]{\left[\begin{array}{#1} #2 \end{array}\right]}
\newcommand{\mycase}[1]{\left\{\begin{array}{ll} #1 \end{array} \right.}
\newcommand{\mb}[1]{\mbox{\boldmath$#1$}}
\newcommand{\lw}[1]{\smash{\lower2.ex\hbox{#1}}}
\newcommand{\zero}{\mathbf{0}}
\newcommand{\one}{\mathbf{1}}
\newcommand{\eps}{\varepsilon}

%%%%%%%%%%%%%%%%%%%%%%%%%%%%%%%%%%%%%%%%%%%%%%%%%%%%%%%%%%%%%%%%%%%%%%%%%%%
% colors
%%%%%%%%%%%%%%%%%%%%%%%%%%%%%%%%%%%%%%%%%%%%%%%%%%%%%%%%%%%%%%%%%%%%%%%%%%%
\newcommand{\myred}[1]{\textcolor{red}{#1}}
\newcommand{\myredbf}[1]{\textcolor{red}{\bf #1}}
\newcommand{\myblue}[1]{\textcolor{blue}{#1}}
\newcommand{\mybluebf}[1]{\textcolor{blue}{\bf #1}}
\newcommand{\mydarkblue}[1]{\textcolor[rgb]{0.0,0.0,0.5}{#1}}
\newcommand{\mygreen}[1]{\textcolor[rgb]{0.0,0.5,0.0}{#1}}
\newcommand{\mygreenbf}[1]{\textcolor[rgb]{0.0,0.5,0.0}{\bf #1}}
\newcommand{\mypurple}[1]{\textcolor[rgb]{0.5,0.0,0.5}{#1}}
\newcommand{\mypurplebf}[1]{\textcolor[rgb]{0.5,0.0,0.5}{\bf #1}}

\begin{document}
%%%%%%%%%%%%%%%%%%%%%%%%%%%%%%%%%%%%%%%%%%%%%%%%%%%%%%%%%%%%%%%%%%%%%%%%%%%
% ここからがレポートの記述
%%%%%%%%%%%%%%%%%%%%%%%%%%%%%%%%%%%%%%%%%%%%%%%%%%%%%%%%%%%%%%%%%%%%%%%%%%%

\begin{center} 
{\large \bf 知能プログラミング演習I 第7回レポート}
\end{center} %

\begin{flushright} 
2025年7月27日 % Date
\hskip 1mm
学籍番号 35714121% 学籍番号
\hskip 1mm
氏名 福富隆大% 氏名
\end{flushright} % Name

\renewcommand{\theenumi}{(\alph{enumi})}

%%%%%%%%%%%%%%%%%%%%%%%%%%%%%%%%%%%%%%%%%%%%%%%%%%%%%%%%%%%%%%%%%%%%%%%%%%%
\section*{課題1}
%%%%%%%%%%%%%%%%%%%%%%%%%%%%%%%%%%%%%%%%%%%%%%%%%%%%%%%%%%%%%%%%%%%%%%%%%%%

\begin{enumerate}
 \item 
	\begin{align*}
	 \text{\tt x} + \text{\tt y}
	 : 
	 \begin{bmatrix}
	  x_1 + y_1 \\
	  x_2 + y_2 \\
	  x_3 + y_3 
	 \end{bmatrix}, \
	 \text{\tt x} - \text{\tt y} &:
	 \begin{bmatrix}
	  x_1 - y_1 \\
	  x_2 - y_2 \\
	  x_3 - y_3 
	 \end{bmatrix}, \
	 \text{\tt x} * \text{\tt y}
	 :
	 \begin{bmatrix}
	  x_1 \cdot y_1 \\
	  x_2 \cdot y_2 \\
	  x_3 \cdot y_3 
	 \end{bmatrix}, \ 
	 \text{\tt x} / \text{\tt y}
	 :
	 \begin{bmatrix}
	  x_1 / y_1 \\
	  x_2 / y_2 \\
	  x_3 / y_3 
	 \end{bmatrix}
	\end{align*}

 \item 
	\begin{align*}
	 \text{\tt M} + \text{\tt x}:
	 \begin{bmatrix}
	  \bm m_1^{\top} + \bm x^{\top} \\
	  \bm m_2^{\top} + \bm x^{\top}
	 \end{bmatrix}, \
	 \text{\tt M} - \text{\tt x}: 
	 \begin{bmatrix}
	  \bm m_1^{\top} - \bm x^{\top} \\
	  \bm m_2^{\top} - \bm x^{\top}
	 \end{bmatrix}, \
	 \text{\tt M} * \text{\tt x}: 
	 \begin{bmatrix}
	  \bm m_1^{\top} \odot \bm x^{\top} \\
	  \bm m_2^{\top} \odot \bm x^{\top}
	 \end{bmatrix}, \
	 \text{\tt M} / \text{\tt x}: 
	 \begin{bmatrix}
	  \bm m_1^{\top} \oslash \bm x^{\top} \\
	  \bm m_2^{\top} \oslash \bm x^{\top}
	 \end{bmatrix}
	\end{align*}
	
 \item 
	\begin{align*}
	 \text{\tt N + K}:  
	 \left[
	 N_1 + K, 
	 N_2 + K
	 \right] 
	 \in \mathbb{R}^{2 \times 4 \times 3}	 
	\end{align*}

 \item 
	\begin{align*}
	 \text{\tt N @ M.transpose(0,1)}
	 &: 
	 \left[
	 N_1 M^{\top}, 
	 N_2 M^{\top}
	 \right] 
	 \in \mathbb{R}^{2 \times 4 \times 2}	 	 
	\end{align*}

	\begin{align*}
	 \text{\tt linear.weight}: \bm W \in \mathbb{R}^{2 \times 3}
	\end{align*}

	\begin{align*}
	 \text{\tt linear(M)} &: 
	 MW^{\top}
	 \in \mathbb{R}^{2 \times 2}
	 \\
	 \text{\tt linear(N)} &:  
	 	 \left[
	 N_1 W^{\top},
	 N_2 W^{\top}
	 \right]
	 \in \mathbb{R}^{2 \times 4 \times 2}
	\end{align*}

 \item 
	\begin{align*}
	 \text{\tt A @ V}: &
	 \left[
	 A_1 V_1,
	 A_2 V_2
	 \right]
	 \in \mathbb{R}^{2 \times 4 \times 3}
	\end{align*}

 \item 
	\begin{align*}
	 \text{\tt V.transpose(1,2))}: &
	 \left[
	 V_1^{\top},
	 V_2^{\top}
	 \right]	 
	 \\
	 \text{\tt N @ V.transpose(1,2)}: &
	 \left[
	 N_1 V_1^{\top},
	 N_2 V_2^{\top}
	 \right]
	 \in \mathbb{R}^{2 \times 4 \times 5}
	\end{align*}

 \item F.softmax関数のdim引数について:
	\begin{itemize}
		\item dim=0の場合:バッチ次元(最初の次元)に対してsoftmaxが適用され、各位置で2つのバッチ間の確率分布が計算される。
		\item dim=1の場合:長さL次元(2番目の次元)に対してsoftmaxが適用され、各系列の各位置間で確率分布が計算される。
		\item dim=2の場合:特徴次元(最後の次元)に対してsoftmaxが適用され、各位置の特徴間で確率分布が計算される。
	\end{itemize} 
\end{enumerate}

%%%%%%%%%%%%%%%%%%%%%%%%%%%%%%%%%%%%%%%%%%%%%%%%%%%%%%%%%%%%%%%%%%%%%%%%%%%
\section*{課題2}
%%%%%%%%%%%%%%%%%%%%%%%%%%%%%%%%%%%%%%%%%%%%%%%%%%%%%%%%%%%%%%%%%%%%%%%%%%%

\begin{enumerate}
 \item tensor配列input\_idxは3つの文字列("tiktok", "tictac", "tuktuk")をトークンIDの配列として表現している。配列のサイズは$3 \times 6$であり、1次元目がバッチサイズ(文字列の数)、2次元目が文字列の長さに対応している。
 
 \item token\_embedding = nn.Embedding(char\_size, dim\_embed)は文字のIDを高次元ベクトルに変換する埋め込み層である。char\_sizeは文字の種類数(26)、dim\_embedは埋め込み次元数(2)を表す。x = token\_embedding(input\_idx)で作成されるxのサイズは$3 \times 6 \times 2$であり、1次元目がバッチサイズ、2次元目が文字列の長さ、3次元目が埋め込み次元に対応している。
 
 \item embed-1.pdfの図は、Token Embeddingによって3つの文字列が2次元空間にプロットされた結果を示している。重複して同じ位置にある点があるのは、同じ文字は常に同じ埋め込みベクトルを持つためである。例えば、全ての't'文字は同じ座標にマップされ、'k'文字も同様に重複している。
 
 \item PositionalEncodingクラスのpeに以下のコードを実装した:
	\begin{verbatim}
	for pos in range(L):
	    for i in range(0, dim_embed, 2):
	        pe[pos, i] = torch.sin(torch.tensor(pos / (10000 ** (2 * i / dim_embed))))
	        if i + 1 < dim_embed:
	            pe[pos, i + 1] = torch.cos(torch.tensor(pos / (10000 ** (2 * i / dim_embed))))
	\end{verbatim}
	
	結果として作成されるembed-2.pdfの図では、Positional Encodingによって位置情報が加えられ、embed-1.pdfで重複していた点が分離されている。同じ文字でも位置によって異なる座標に配置されるようになった。
 
 \item SelfAttentionクラスのforward関数に以下のコードを実装した:
	\begin{verbatim}
	K_t = K.transpose(-2, -1)
	scaled_dot_product = Q @ K_t / torch.sqrt(torch.tensor(self.head_size, dtype=torch.float32))
	atten_wei = F.softmax(scaled_dot_product, dim=-1)
	out = atten_wei @ V
	\end{verbatim}
	
	結果として作成されるembed-3.pdfの図では、Self-Attentionによって各位置の表現が他の位置との関係を考慮して更新され、embed-2.pdfからさらに点の配置が変化している。文脈を考慮した表現になったことで、重複の度合いがさらに変化した。
\end{enumerate}

%%%%%%%%%%%%%%%%%%%%%%%%%%%%%%%%%%%%%%%%%%%%%%%%%%%%%%%%%%%%%%%%%%%%%%%%%%%
\section*{課題3}
%%%%%%%%%%%%%%%%%%%%%%%%%%%%%%%%%%%%%%%%%%%%%%%%%%%%%%%%%%%%%%%%%%%%%%%%%%%

\begin{enumerate}
 \item loss\_fn(logits, y)は交差エントロピー損失を計算している。logitsはモデルの出力(各文字の予測スコア)で、yは正解の文字IDである。この損失関数により、モデルは次の文字を正確に予測するように学習される。

 \item Transformerクラスを以下のアーキテクチャで完成させた:
	\begin{itemize}
		\item Token埋め込み → Positional encoding → Self-attention (single head) → 全結合層
		\item 全結合層:線形層(output: 100次元) → ReLU → 線形層(output: char\_size次元)
	\end{itemize}
	
	学習後の損失推移はloss.pdfに示されており、5000ステップの学習で損失が約4.5から0.2まで減少した。
	
	学習後に生成された文章の例:
	\begin{enumerate}
		\item 入力「ももからうまれた」→「ももからうまれたおとこのこを、おじいさんがはたけではたらいていますと、「やーい、よぼよぼじじい。よぼよぼじじい」と、」
		\item 入力「ももからうまれた」→「ももからうまれたおとこのこを、おじいさんがはたけではたわへせかい」すずめをつか」おおばあさんはかにこのってかえってい」
		\item 入力「ももからうまれた」→「ももからうまれたおとこのこを、おじいさんがはたけではたらいていますと、「やーい、よぼよぼじじい。よぼよぼじじい」と、」
		\item (学習前の例)「ももからうまれたおずぽでごはばがはそりっ!まびけ「つぽとえら3たらなも2『ねー「きわとゅぎご・て・え」
		\item (学習後の特徴)昔話らしい文体で、文法的に意味のある日本語が生成されるようになった
	\end{enumerate}
\end{enumerate}

\end{document}
