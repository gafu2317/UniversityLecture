\documentclass{jarticle}

%%%%%%%%%%%%%%%%%%%%%%%%%%%%%%%%%%%%%%%%%%%%%%%%%%%%%%%%%%%%%%%%%%%%%%%%%%%
% packages
%%%%%%%%%%%%%%%%%%%%%%%%%%%%%%%%%%%%%%%%%%%%%%%%%%%%%%%%%%%%%%%%%%%%%%%%%%%
\usepackage{amsmath}
\usepackage{amssymb}
\usepackage{amsfonts}
\usepackage[dvipdfmx]{graphicx}
\usepackage[dvipdfmx]{color}
\usepackage{graphicx}
\usepackage{bm}

%%%%%%%%%%%%%%%%%%%%%%%%%%%%%%%%%%%%%%%%%%%%%%%%%%%%%%%%%%%%%%%%%%%%%%%%%%%
% format stuffs
%%%%%%%%%%%%%%%%%%%%%%%%%%%%%%%%%%%%%%%%%%%%%%%%%%%%%%%%%%%%%%%%%%%%%%%%%%%
\setlength{\oddsidemargin}{0.455cm} 
\setlength{\evensidemargin}{0.455cm} 
\setlength{\textwidth}{15.5cm} 
\setlength{\textheight}{22.54cm}
\setlength{\headheight}{0mm}
\setlength{\headsep}{0mm}
\setlength{\topskip}{0mm}
\setcounter{topnumber}{100}
\setcounter{bottomnumber}{100}
\setcounter{totalnumber}{100}
\renewcommand{\topfraction}{1.0}
\renewcommand{\bottomfraction}{1.0}
\renewcommand{\textfraction}{0.0}
\renewcommand{\floatpagefraction}{0.0}
\renewcommand{\baselinestretch}{1.0}
\pagestyle{empty}

%%%%%%%%%%%%%%%%%%%%%%%%%%%%%%%%%%%%%%%%%%%%%%%%%%%%%%%%%%%%%%%%%%%%%%%%%%%
% math symbols and commands
%%%%%%%%%%%%%%%%%%%%%%%%%%%%%%%%%%%%%%%%%%%%%%%%%%%%%%%%%%%%%%%%%%%%%%%%%%%
\newcommand{\eq}[1]{(\ref{#1})}
\newcommand{\mtx}[2]{\left[\begin{array}{#1} #2 \end{array}\right]}
\newcommand{\mycase}[1]{\left\{\begin{array}{ll} #1 \end{array} \right.}
\newcommand{\mb}[1]{\mbox{\boldmath$#1$}}
\newcommand{\lw}[1]{\smash{\lower2.ex\hbox{#1}}}
\newcommand{\zero}{\mathbf{0}}
\newcommand{\one}{\mathbf{1}}
\newcommand{\eps}{\varepsilon}

%%%%%%%%%%%%%%%%%%%%%%%%%%%%%%%%%%%%%%%%%%%%%%%%%%%%%%%%%%%%%%%%%%%%%%%%%%%
% colors
%%%%%%%%%%%%%%%%%%%%%%%%%%%%%%%%%%%%%%%%%%%%%%%%%%%%%%%%%%%%%%%%%%%%%%%%%%%
\newcommand{\myred}[1]{\textcolor{red}{#1}}
\newcommand{\myredbf}[1]{\textcolor{red}{\bf #1}}
\newcommand{\myblue}[1]{\textcolor{blue}{#1}}
\newcommand{\mybluebf}[1]{\textcolor{blue}{\bf #1}}
\newcommand{\mydarkblue}[1]{\textcolor[rgb]{0.0,0.0,0.5}{#1}}
\newcommand{\mygreen}[1]{\textcolor[rgb]{0.0,0.5,0.0}{#1}}
\newcommand{\mygreenbf}[1]{\textcolor[rgb]{0.0,0.5,0.0}{\bf #1}}
\newcommand{\mypurple}[1]{\textcolor[rgb]{0.5,0.0,0.5}{#1}}
\newcommand{\mypurplebf}[1]{\textcolor[rgb]{0.5,0.0,0.5}{\bf #1}}

\begin{document}
%%%%%%%%%%%%%%%%%%%%%%%%%%%%%%%%%%%%%%%%%%%%%%%%%%%%%%%%%%%%%%%%%%%%%%%%%%%
% ここからがレポートの記述
%%%%%%%%%%%%%%%%%%%%%%%%%%%%%%%%%%%%%%%%%%%%%%%%%%%%%%%%%%%%%%%%%%%%%%%%%%%

\begin{center} 
{\large \bf 知能プログラミング演習I 第7回レポート}
\end{center} %

\begin{flushright} 
XXXX年YYYY月ZZZZ日 % Date
\hskip 1mm
学籍番号 % 学籍番号
\hskip 1mm
氏名 % 氏名
\end{flushright} % Name

\renewcommand{\theenumi}{(\alph{enumi})}

%%%%%%%%%%%%%%%%%%%%%%%%%%%%%%%%%%%%%%%%%%%%%%%%%%%%%%%%%%%%%%%%%%%%%%%%%%%
\section*{課題1}
%%%%%%%%%%%%%%%%%%%%%%%%%%%%%%%%%%%%%%%%%%%%%%%%%%%%%%%%%%%%%%%%%%%%%%%%%%%

\begin{enumerate}
 \item 
	\begin{align*}
	 \text{\tt x} + \text{\tt y}
	 : 
	 \begin{bmatrix}
	  x_1 + y_1 \\
	  x_2 + y_2 \\
	  x_3 + y_3 
	 \end{bmatrix}, \
	 \text{\tt x} - \text{\tt y} &:
	 \begin{bmatrix}
	  ??? \\
	  ??? \\
	  ??? 
	 \end{bmatrix}, \
	 \text{\tt x} * \text{\tt y}
	 :
	 \begin{bmatrix}
	  ??? \\
	  ??? \\
	  ??? 
	 \end{bmatrix}, \ 
	 \text{\tt x} / \text{\tt y}
	 :
	 \begin{bmatrix}
	  ??? \\
	  ??? \\
	  ??? 
	 \end{bmatrix}
	\end{align*}

 \item 
	\begin{align*}
	 \text{\tt M} + \text{\tt x}:
	 \begin{bmatrix}
	  ??? \\
	  ??? %\\
%	  ??? 
	 \end{bmatrix}, \
	 % \\
	 \text{\tt M} - \text{\tt x}: 
	 \begin{bmatrix}
	  ??? \\
	  ??? %\\
%	  ??? 
	 \end{bmatrix}, \
	 % \\
	 \text{\tt M} * \text{\tt x}: 
	 \begin{bmatrix}
	  ??? \\
	  ??? %\\
%	  ??? 
	 \end{bmatrix}, \
	 % \\
	 \text{\tt M} / \text{\tt x}: 
	 \begin{bmatrix}
	  ??? \\
	  ??? %\\
%	  ??? 
	 \end{bmatrix}
	\end{align*}
	
 \item 
	\begin{align*}
	 \text{\tt N + K}:  
	 \left[
	 ???, 
	 ???
	 \right] 
	 \in \mathbb{R}^{??? \times ??? \times ???}	 
	\end{align*}

 \item 
	\begin{align*}
	 \text{\tt N @ M.transpose(0,1)}
	 &: 
	 % =
	 \left[
	 ???, 
	 ???
	 \right] 
	 \in \mathbb{R}^{??? \times ??? \times ???}	 	 
	\end{align*}

	\begin{align*}
	 \text{\tt linear.weight}: \bm W \in \mathbb{R}^{??? \times ???}
	\end{align*}

	\begin{align*}
	 \text{\tt linear(M)} &: 
	 ???
	 \in \mathbb{R}^{??? \times ???}
	 \\
	 \text{\tt linear(N)} &:  
	 	 \left[
	 ???,
	 ???
	 \right]
	 \in \mathbb{R}^{??? \times ??? \times ???}
	\end{align*}

 \item 
	\begin{align*}
	 \text{\tt A @ V}: &
	 \left[
	 ???,
	 ???
	 \right]
	 \in \mathbb{R}^{??? \times ??? \times ???}
	\end{align*}

 \item 
	\begin{align*}
	 \text{\tt V.transpose(1,2))}: &
	 \left[
	 ???,
	 ???
	 \right]	 
	 \\
	 \text{\tt N @ V.transpose(1,2)}: &
	 \left[
	 ???,
	 ???
	 \right]
	 \in \mathbb{R}^{??? \times ??? \times ???}
	\end{align*}

 \item 
\end{enumerate}

%%%%%%%%%%%%%%%%%%%%%%%%%%%%%%%%%%%%%%%%%%%%%%%%%%%%%%%%%%%%%%%%%%%%%%%%%%%
\section*{課題2}
%%%%%%%%%%%%%%%%%%%%%%%%%%%%%%%%%%%%%%%%%%%%%%%%%%%%%%%%%%%%%%%%%%%%%%%%%%%

\begin{enumerate}
 \item 
 \item 
 \item 
 \item 
 \item 
\end{enumerate}

%%%%%%%%%%%%%%%%%%%%%%%%%%%%%%%%%%%%%%%%%%%%%%%%%%%%%%%%%%%%%%%%%%%%%%%%%%%
\section*{課題3}
%%%%%%%%%%%%%%%%%%%%%%%%%%%%%%%%%%%%%%%%%%%%%%%%%%%%%%%%%%%%%%%%%%%%%%%%%%%

\begin{enumerate}
 \item 
 \item 
\end{enumerate}

\end{document}
