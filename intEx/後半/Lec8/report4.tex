\documentclass{jarticle}

%%%%%%%%%%%%%%%%%%%%%%%%%%%%%%%%%%%%%%%%%%%%%%%%%%%%%%%%%%%%%%%%%%%%%%%%%%%
% packages
%%%%%%%%%%%%%%%%%%%%%%%%%%%%%%%%%%%%%%%%%%%%%%%%%%%%%%%%%%%%%%%%%%%%%%%%%%%
\usepackage{amsmath}
\usepackage{amssymb}
\usepackage{amsfonts}
\usepackage[dvipdfmx]{graphicx}
\usepackage[dvipdfmx]{color}
\usepackage{graphicx}
\usepackage{bm}
\usepackage{url}

%%%%%%%%%%%%%%%%%%%%%%%%%%%%%%%%%%%%%%%%%%%%%%%%%%%%%%%%%%%%%%%%%%%%%%%%%%%
% format stuffs
%%%%%%%%%%%%%%%%%%%%%%%%%%%%%%%%%%%%%%%%%%%%%%%%%%%%%%%%%%%%%%%%%%%%%%%%%%%
\setlength{\oddsidemargin}{0.455cm} 
\setlength{\evensidemargin}{0.455cm} 
\setlength{\textwidth}{15.5cm} 
\setlength{\textheight}{22.54cm}
\setlength{\headheight}{0mm}
\setlength{\headsep}{0mm}
\setlength{\topskip}{0mm}
\setcounter{topnumber}{100}
\setcounter{bottomnumber}{100}
\setcounter{totalnumber}{100}
\renewcommand{\topfraction}{1.0}
\renewcommand{\bottomfraction}{1.0}
\renewcommand{\textfraction}{0.0}
\renewcommand{\floatpagefraction}{0.0}
\renewcommand{\baselinestretch}{1.0}
\pagestyle{empty}

%%%%%%%%%%%%%%%%%%%%%%%%%%%%%%%%%%%%%%%%%%%%%%%%%%%%%%%%%%%%%%%%%%%%%%%%%%%
% math symbols and commands
%%%%%%%%%%%%%%%%%%%%%%%%%%%%%%%%%%%%%%%%%%%%%%%%%%%%%%%%%%%%%%%%%%%%%%%%%%%
\newcommand{\eq}[1]{(\ref{#1})}
\newcommand{\mtx}[2]{\left[\begin{array}{#1} #2 \end{array}\right]}
\newcommand{\mycase}[1]{\left\{\begin{array}{ll} #1 \end{array} \right.}
\newcommand{\mb}[1]{\mbox{\boldmath$#1$}}
\newcommand{\lw}[1]{\smash{\lower2.ex\hbox{#1}}}
\newcommand{\zero}{\mathbf{0}}
\newcommand{\one}{\mathbf{1}}
\newcommand{\eps}{\varepsilon}

%%%%%%%%%%%%%%%%%%%%%%%%%%%%%%%%%%%%%%%%%%%%%%%%%%%%%%%%%%%%%%%%%%%%%%%%%%%
% colors
%%%%%%%%%%%%%%%%%%%%%%%%%%%%%%%%%%%%%%%%%%%%%%%%%%%%%%%%%%%%%%%%%%%%%%%%%%%
\newcommand{\myred}[1]{\textcolor{red}{#1}}
\newcommand{\myredbf}[1]{\textcolor{red}{\bf #1}}
\newcommand{\myblue}[1]{\textcolor{blue}{#1}}
\newcommand{\mybluebf}[1]{\textcolor{blue}{\bf #1}}
\newcommand{\mydarkblue}[1]{\textcolor[rgb]{0.0,0.0,0.5}{#1}}
\newcommand{\mygreen}[1]{\textcolor[rgb]{0.0,0.5,0.0}{#1}}
\newcommand{\mygreenbf}[1]{\textcolor[rgb]{0.0,0.5,0.0}{\bf #1}}
\newcommand{\mypurple}[1]{\textcolor[rgb]{0.5,0.0,0.5}{#1}}
\newcommand{\mypurplebf}[1]{\textcolor[rgb]{0.5,0.0,0.5}{\bf #1}}

\begin{document}
%%%%%%%%%%%%%%%%%%%%%%%%%%%%%%%%%%%%%%%%%%%%%%%%%%%%%%%%%%%%%%%%%%%%%%%%%%%
% ここから課題レポートの記述
%%%%%%%%%%%%%%%%%%%%%%%%%%%%%%%%%%%%%%%%%%%%%%%%%%%%%%%%%%%%%%%%%%%%%%%%%%%

\begin{center} 
{\large \bf 知的プログラミング演習I 第8回レポート}
\end{center} %

\begin{flushright} 
2025年7月27日 % Date
\hskip 1mm
学籍番号 35714121% 学籍番号
\hskip 1mm
氏名福富隆大 % 氏名
\end{flushright} % Name

%%%%%%%%%%%%%%%%%%%%%%%%%%%%%%%%%%%%%%%%%%%%%%%%%%%%%%%%%%%%%%%%%%%%%%%%%%%
\section{レポートのテーマ}
%%%%%%%%%%%%%%%%%%%%%%%%%%%%%%%%%%%%%%%%%%%%%%%%%%%%%%%%%%%%%%%%%%%%%%%%%%%

本レポートでは、課題選択肢の中から「複雑なニューラルネットワークの実装」をテーマとして選択する。LSTM(Long Short-Term Memory)を用いた再帰型ニューラルネットワークを実装し、時系列データの分類問題に適用した結果について報告する。

%%%%%%%%%%%%%%%%%%%%%%%%%%%%%%%%%%%%%%%%%%%%%%%%%%%%%%%%%%%%%%%%%%%%%%%%%%%
\section{実装したモデル}
%%%%%%%%%%%%%%%%%%%%%%%%%%%%%%%%%%%%%%%%%%%%%%%%%%%%%%%%%%%%%%%%%%%%%%%%%%%

今回実装したのは、LSTMを用いた再帰型ニューラルネットワークである。LSTMは通常のRNN(再帰型ニューラルネットワーク)の欠点である勾配消失問題を解決するために開発された手法で、長期の依存関係を学習できる特徴がある。

LSTMの主要な構成要素は以下の通り:
\begin{itemize}
\item \textbf{忘却ゲート}: 過去の情報をどれだけ忘れるかを制御
\item \textbf{入力ゲート}: 新しい情報をどれだけ取り入れるかを制御
\item \textbf{出力ゲート}: 出力をどれだけ制御するか
\item \textbf{セル状態}: 長期記憶を保持する内部状態
\end{itemize}

今回の実装では、時系列データから特定のパターン([1,0,1]の並び)を検出する2クラス分類問題を解くことを目標とした。

%%%%%%%%%%%%%%%%%%%%%%%%%%%%%%%%%%%%%%%%%%%%%%%%%%%%%%%%%%%%%%%%%%%%%%%%%%%
\section{ネットワークの詳細設定}
%%%%%%%%%%%%%%%%%%%%%%%%%%%%%%%%%%%%%%%%%%%%%%%%%%%%%%%%%%%%%%%%%%%%%%%%%%%

実装したLSTMネットワークの詳細設定は以下の通り:

\begin{itemize}
\item \textbf{入力層}: 2次元(0と1のワンホットエンコーディング)
\item \textbf{隠れ層}: 10ユニット(LSTMセル)
\item \textbf{出力層}: 2次元(パターンあり/なしの2クラス)
\item \textbf{系列長}: 15ステップ
\item \textbf{活性化関数}: シグモイド関数(ゲート)、tanh関数(セル状態)
\item \textbf{重み初期化}: 標準偏差0.1の正規分布
\item \textbf{サンプル数}: 200個のランダム生成データ
\end{itemize}

プログラムは lec8.py に保存されており、完全なLSTMの順伝播計算を含んでいる。

%%%%%%%%%%%%%%%%%%%%%%%%%%%%%%%%%%%%%%%%%%%%%%%%%%%%%%%%%%%%%%%%%%%%%%%%%%%
\section{実験結果}
%%%%%%%%%%%%%%%%%%%%%%%%%%%%%%%%%%%%%%%%%%%%%%%%%%%%%%%%%%%%%%%%%%%%%%%%%%%

実装したLSTMモデルでいくつかのサンプルデータに対して予測を実行した。ただし、今回は学習処理(逆伝播による重み更新)は実装していないため、ランダム初期化状態での予測結果となっている。

実験では以下のような結果が得られた:
\begin{itemize}
\item モデルは正常に動作し、各時系列に対して確率値を出力
\item 初期状態では予測精度は約50\%(ランダム予測レベル)
\item LSTMの内部状態(隠れ状態とセル状態)が正しく更新されることを確認
\item 異なる入力系列に対して異なる出力を生成することを確認
\end{itemize}

実際の学習を行うためには、損失関数の計算、逆伝播アルゴリズム、最適化手法の実装が必要である。

%%%%%%%%%%%%%%%%%%%%%%%%%%%%%%%%%%%%%%%%%%%%%%%%%%%%%%%%%%%%%%%%%%%%%%%%%%%
\section{考察}
%%%%%%%%%%%%%%%%%%%%%%%%%%%%%%%%%%%%%%%%%%%%%%%%%%%%%%%%%%%%%%%%%%%%%%%%%%%

LSTMの実装を通じて、再帰型ニューラルネットワークの複雑さと有用性を理解できた。特に、ゲート機構による情報の制御が時系列データの処理に重要な役割を果たすことが分かった。

実装上の課題として、以下の点が挙げられる:
\begin{itemize}
\item 勾配の計算が複雑になる(時間方向への逆伝播が必要)
\item パラメータ数が多く、適切な初期化が重要
\item 長い系列では計算コストが高くなる
\end{itemize}

また、実用的な応用を考えると、適切な学習率、正則化、バッチ処理などの技術が必要になる。今回の実装は基本的な構造の理解に重点を置いたが、実際のプロジェクトではより洗練された実装が求められるだろう。

LSTMのような複雑なモデルを一から実装することで、深層学習フレームワークの便利さも再認識できた。

%%%%%%%%%%%%%%%%%%%%%%%%%%%%%%%%%%%%%%%%%%%%%%%%%%%%%%%%%%%%%%%%%%%%%%%%%%%
\section{参考文献}
%%%%%%%%%%%%%%%%%%%%%%%%%%%%%%%%%%%%%%%%%%%%%%%%%%%%%%%%%%%%%%%%%%%%%%%%%%%

\begin{enumerate}
\item LSTMの実装(RNN・自然言語処理) \url{https://qiita.com/hara_tatsu/items/c3ba100e95e600846125}
\item 機械学習の学習法 \url{https://qiita.com/hokkey621/items/404e04d6057b98128971}
\item ニューラルネットワークの基礎 \url{https://tutorials.chainer.org/ja/13_Basics_of_Neural_Networks.html}
\item LSTM and RNN Tutorial \url{https://github.com/omerbsezer/LSTM_RNN_Tutorials_with_Demo}
\item [PyTorch]RNNを使った時系列予測 \url{https://qiita.com/kuk_a_i_ai/items/0ea4b93d767ce7c83145}
\end{enumerate}

%%%%%%%%%%%%%%%%%%%%%%%%%%%%%%%%%%%%%%%%%%%%%%%%%%%%%%%%%%%%%%%%%%%%%%%%%%%
% ここまで課題レポートの記述
%%%%%%%%%%%%%%%%%%%%%%%%%%%%%%%%%%%%%%%%%%%%%%%%%%%%%%%%%%%%%%%%%%%%%%%%%%%
\end{document}