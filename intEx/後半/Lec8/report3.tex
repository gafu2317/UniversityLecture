\documentclass{jarticle}

%%%%%%%%%%%%%%%%%%%%%%%%%%%%%%%%%%%%%%%%%%%%%%%%%%%%%%%%%%%%%%%%%%%%%%%%%%%
% packages
%%%%%%%%%%%%%%%%%%%%%%%%%%%%%%%%%%%%%%%%%%%%%%%%%%%%%%%%%%%%%%%%%%%%%%%%%%%
\usepackage{amsmath}
\usepackage{amssymb}
\usepackage{amsfonts}
\usepackage[dvipdfmx]{graphicx}
\usepackage[dvipdfmx]{color}
\usepackage{graphicx}
\usepackage{bm}
\usepackage{url}

%%%%%%%%%%%%%%%%%%%%%%%%%%%%%%%%%%%%%%%%%%%%%%%%%%%%%%%%%%%%%%%%%%%%%%%%%%%
% format stuffs
%%%%%%%%%%%%%%%%%%%%%%%%%%%%%%%%%%%%%%%%%%%%%%%%%%%%%%%%%%%%%%%%%%%%%%%%%%%
\setlength{\oddsidemargin}{0.455cm} 
\setlength{\evensidemargin}{0.455cm} 
\setlength{\textwidth}{15.5cm} 
\setlength{\textheight}{22.54cm}
\setlength{\headheight}{0mm}
\setlength{\headsep}{0mm}
\setlength{\topskip}{0mm}
\setcounter{topnumber}{100}
\setcounter{bottomnumber}{100}
\setcounter{totalnumber}{100}
\renewcommand{\topfraction}{1.0}
\renewcommand{\bottomfraction}{1.0}
\renewcommand{\textfraction}{0.0}
\renewcommand{\floatpagefraction}{0.0}
\renewcommand{\baselinestretch}{1.0}
\pagestyle{empty}

%%%%%%%%%%%%%%%%%%%%%%%%%%%%%%%%%%%%%%%%%%%%%%%%%%%%%%%%%%%%%%%%%%%%%%%%%%%
% math symbols and commands
%%%%%%%%%%%%%%%%%%%%%%%%%%%%%%%%%%%%%%%%%%%%%%%%%%%%%%%%%%%%%%%%%%%%%%%%%%%
\newcommand{\eq}[1]{(\ref{#1})}
\newcommand{\mtx}[2]{\left[\begin{array}{#1} #2 \end{array}\right]}
\newcommand{\mycase}[1]{\left\{\begin{array}{ll} #1 \end{array} \right.}
\newcommand{\mb}[1]{\mbox{\boldmath$#1$}}
\newcommand{\lw}[1]{\smash{\lower2.ex\hbox{#1}}}
\newcommand{\zero}{\mathbf{0}}
\newcommand{\one}{\mathbf{1}}
\newcommand{\eps}{\varepsilon}

%%%%%%%%%%%%%%%%%%%%%%%%%%%%%%%%%%%%%%%%%%%%%%%%%%%%%%%%%%%%%%%%%%%%%%%%%%%
% colors
%%%%%%%%%%%%%%%%%%%%%%%%%%%%%%%%%%%%%%%%%%%%%%%%%%%%%%%%%%%%%%%%%%%%%%%%%%%
\newcommand{\myred}[1]{\textcolor{red}{#1}}
\newcommand{\myredbf}[1]{\textcolor{red}{\bf #1}}
\newcommand{\myblue}[1]{\textcolor{blue}{#1}}
\newcommand{\mybluebf}[1]{\textcolor{blue}{\bf #1}}
\newcommand{\mydarkblue}[1]{\textcolor[rgb]{0.0,0.0,0.5}{#1}}
\newcommand{\mygreen}[1]{\textcolor[rgb]{0.0,0.5,0.0}{#1}}
\newcommand{\mygreenbf}[1]{\textcolor[rgb]{0.0,0.5,0.0}{\bf #1}}
\newcommand{\mypurple}[1]{\textcolor[rgb]{0.5,0.0,0.5}{#1}}
\newcommand{\mypurplebf}[1]{\textcolor[rgb]{0.5,0.0,0.5}{\bf #1}}

\begin{document}
%%%%%%%%%%%%%%%%%%%%%%%%%%%%%%%%%%%%%%%%%%%%%%%%%%%%%%%%%%%%%%%%%%%%%%%%%%%
% ここから課題レポートの記述
%%%%%%%%%%%%%%%%%%%%%%%%%%%%%%%%%%%%%%%%%%%%%%%%%%%%%%%%%%%%%%%%%%%%%%%%%%%

\begin{center} 
{\large \bf 知的プログラミング演習I 第8回レポート}
\end{center} %

\begin{flushright} 
2025年7月27日 % Date
\hskip 1mm
学籍番号 35714121% 学籍番号
\hskip 1mm
氏名福富隆大 % 氏名
\end{flushright} % Name

%%%%%%%%%%%%%%%%%%%%%%%%%%%%%%%%%%%%%%%%%%%%%%%%%%%%%%%%%%%%%%%%%%%%%%%%%%%
\section{レポートのテーマ}
%%%%%%%%%%%%%%%%%%%%%%%%%%%%%%%%%%%%%%%%%%%%%%%%%%%%%%%%%%%%%%%%%%%%%%%%%%%

本レポートでは、課題選択肢の中から「深層学習の理論」をテーマとして選択する。深層学習がなぜ高い性能を発揮するのか、どのような理論的根拠があるのか、また現在も未解決な問題について調査し、考察する。

%%%%%%%%%%%%%%%%%%%%%%%%%%%%%%%%%%%%%%%%%%%%%%%%%%%%%%%%%%%%%%%%%%%%%%%%%%%
\section{調査した深層学習の理論的側面}
%%%%%%%%%%%%%%%%%%%%%%%%%%%%%%%%%%%%%%%%%%%%%%%%%%%%%%%%%%%%%%%%%%%%%%%%%%%

\subsection{万能近似定理}
深層学習の理論的基盤の一つに万能近似定理がある。この定理は、十分な数の隠れ層を持つニューラルネットワークが、任意の連続関数を任意の精度で近似できることを示している。これが深層学習の表現力の高さを理論的に裏付けている。

\subsection{表現学習理論}
深層学習は、データから自動的に有用な特徴を学習する「表現学習」を行う。浅い層では基本的な特徴を捉え、深い層ではより抽象的で複雑な特徴を学習する。この階層的な特徴学習により、従来の手法では困難だった複雑なパターン認識が可能になった。

\subsection{勾配降下法と最適化理論}
深層学習の学習過程は勾配降下法に基づいている。しかし、非凸最適化問題であるため、理論的には局所最適解に陥る可能性がある。現実的には、高次元空間では局所最適解の多くが大域最適解に近い性能を示すという研究結果がある。

%%%%%%%%%%%%%%%%%%%%%%%%%%%%%%%%%%%%%%%%%%%%%%%%%%%%%%%%%%%%%%%%%%%%%%%%%%%
\section{理論的な結果と設定}
%%%%%%%%%%%%%%%%%%%%%%%%%%%%%%%%%%%%%%%%%%%%%%%%%%%%%%%%%%%%%%%%%%%%%%%%%%%

現在明らかになっている理論的結果は以下の通り:

\begin{itemize}
\item \textbf{過学習の理解}: 十分大きなネットワークでも適切な正則化により汎化性能が向上する
\item \textbf{深さの重要性}: 深いネットワークは浅いネットワークより指数関数的に少ないパラメータで同等の表現力を持つ
\item \textbf{初期化の影響}: 重みの初期化方法が学習の成功に大きく影響する
\item \textbf{バッチ正規化の効果}: 内部共変量シフトを減らし、学習を安定化させる
\end{itemize}

これらの結果は、特定の仮定の下で成り立つものが多く、実際の応用では経験的な知見に頼る部分も大きい。

%%%%%%%%%%%%%%%%%%%%%%%%%%%%%%%%%%%%%%%%%%%%%%%%%%%%%%%%%%%%%%%%%%%%%%%%%%%
\section{未解決な問題}
%%%%%%%%%%%%%%%%%%%%%%%%%%%%%%%%%%%%%%%%%%%%%%%%%%%%%%%%%%%%%%%%%%%%%%%%%%%

深層学習にはまだ多くの未解決問題がある:

\begin{itemize}
\item \textbf{汎化能力の謎}: なぜ過度に複雑なモデルでも良い汎化性能を示すのか
\item \textbf{最適化の理論}: 非凸最適化でなぜ良い解が見つかるのか
\item \textbf{解釈性の問題}: ネットワークの内部でどのような処理が行われているか
\item \textbf{学習データ量}: どれくらいのデータがあれば十分な性能が得られるか
\end{itemize}

特に、深層学習モデルがブラックボックス的であることは、医療や金融などの重要な分野での応用において大きな課題となっている。

%%%%%%%%%%%%%%%%%%%%%%%%%%%%%%%%%%%%%%%%%%%%%%%%%%%%%%%%%%%%%%%%%%%%%%%%%%%
\section{考察}
%%%%%%%%%%%%%%%%%%%%%%%%%%%%%%%%%%%%%%%%%%%%%%%%%%%%%%%%%%%%%%%%%%%%%%%%%%%

深層学習の理論研究は実用化に比べて遅れているが、徐々に理解が深まってきている。理論的な裏付けがあることで、より効率的なモデル設計や学習方法の開発が可能になる。

一方で、理論と実践の間にはまだギャップがある。多くの実用的な技術は経験的に発見されており、後から理論的説明がつけられることが多い。これは深層学習分野の特徴的な側面でもある。

個人的には、完全な理論的理解がなくても実用的な価値があることは重要だと思う。しかし、安全性や信頼性が求められる分野では、理論的な保証が必要だろう。

今後は、理論と実践のバランスを取りながら、より堅実で信頼できるAI技術の発展が期待される。

%%%%%%%%%%%%%%%%%%%%%%%%%%%%%%%%%%%%%%%%%%%%%%%%%%%%%%%%%%%%%%%%%%%%%%%%%%%
\section{参考文献}
%%%%%%%%%%%%%%%%%%%%%%%%%%%%%%%%%%%%%%%%%%%%%%%%%%%%%%%%%%%%%%%%%%%%%%%%%%%

\begin{enumerate}
\item 汎用基盤技術研究グループの活動紹介 \url{https://aip.riken.jp/uploads/20210323_AIP_Generic.pdf}
\item ニューラルネットワークの最適化理論 \url{https://orsj.org/wp-content/corsj/or65-12/or65_12_643.pdf}
\item 深層学習の汎化の謎をめぐって \url{http://cometscome.github.io/DLAP2020/slides/DL_Physics2020_mori.pdf}
\item 統計・機械学習の理論を学ぶ手順 \url{https://qiita.com/kueda_cs/items/28008db6491c71ac5659}
\item 深層学習の統計的学習理論 \url{https://www.ieice.org/ess/sita/forum/article/2022/202212191133.pdf}
\end{enumerate}

%%%%%%%%%%%%%%%%%%%%%%%%%%%%%%%%%%%%%%%%%%%%%%%%%%%%%%%%%%%%%%%%%%%%%%%%%%%
% ここまで課題レポートの記述
%%%%%%%%%%%%%%%%%%%%%%%%%%%%%%%%%%%%%%%%%%%%%%%%%%%%%%%%%%%%%%%%%%%%%%%%%%%
\end{document}