\documentclass{jarticle}

%%%%%%%%%%%%%%%%%%%%%%%%%%%%%%%%%%%%%%%%%%%%%%%%%%%%%%%%%%%%%%%%%%%%%%%%%%%
% packages
%%%%%%%%%%%%%%%%%%%%%%%%%%%%%%%%%%%%%%%%%%%%%%%%%%%%%%%%%%%%%%%%%%%%%%%%%%%
\usepackage{amsmath}
\usepackage{amssymb}
\usepackage{amsfonts}
\usepackage[dvipdfmx]{graphicx}
\usepackage[dvipdfmx]{color}
\usepackage{graphicx}
\usepackage{bm}
\usepackage{url}

%%%%%%%%%%%%%%%%%%%%%%%%%%%%%%%%%%%%%%%%%%%%%%%%%%%%%%%%%%%%%%%%%%%%%%%%%%%
% format stuffs
%%%%%%%%%%%%%%%%%%%%%%%%%%%%%%%%%%%%%%%%%%%%%%%%%%%%%%%%%%%%%%%%%%%%%%%%%%%
\setlength{\oddsidemargin}{0.455cm} 
\setlength{\evensidemargin}{0.455cm} 
\setlength{\textwidth}{15.5cm} 
\setlength{\textheight}{22.54cm}
\setlength{\headheight}{0mm}
\setlength{\headsep}{0mm}
\setlength{\topskip}{0mm}
\setcounter{topnumber}{100}
\setcounter{bottomnumber}{100}
\setcounter{totalnumber}{100}
\renewcommand{\topfraction}{1.0}
\renewcommand{\bottomfraction}{1.0}
\renewcommand{\textfraction}{0.0}
\renewcommand{\floatpagefraction}{0.0}
\renewcommand{\baselinestretch}{1.0}
\pagestyle{empty}

%%%%%%%%%%%%%%%%%%%%%%%%%%%%%%%%%%%%%%%%%%%%%%%%%%%%%%%%%%%%%%%%%%%%%%%%%%%
% math symbols and commands
%%%%%%%%%%%%%%%%%%%%%%%%%%%%%%%%%%%%%%%%%%%%%%%%%%%%%%%%%%%%%%%%%%%%%%%%%%%
\newcommand{\eq}[1]{(\ref{#1})}
\newcommand{\mtx}[2]{\left[\begin{array}{#1} #2 \end{array}\right]}
\newcommand{\mycase}[1]{\left\{\begin{array}{ll} #1 \end{array} \right.}
\newcommand{\mb}[1]{\mbox{\boldmath$#1$}}
\newcommand{\lw}[1]{\smash{\lower2.ex\hbox{#1}}}
\newcommand{\zero}{\mathbf{0}}
\newcommand{\one}{\mathbf{1}}
\newcommand{\eps}{\varepsilon}

%%%%%%%%%%%%%%%%%%%%%%%%%%%%%%%%%%%%%%%%%%%%%%%%%%%%%%%%%%%%%%%%%%%%%%%%%%%
% colors
%%%%%%%%%%%%%%%%%%%%%%%%%%%%%%%%%%%%%%%%%%%%%%%%%%%%%%%%%%%%%%%%%%%%%%%%%%%
\newcommand{\myred}[1]{\textcolor{red}{#1}}
\newcommand{\myredbf}[1]{\textcolor{red}{\bf #1}}
\newcommand{\myblue}[1]{\textcolor{blue}{#1}}
\newcommand{\mybluebf}[1]{\textcolor{blue}{\bf #1}}
\newcommand{\mydarkblue}[1]{\textcolor[rgb]{0.0,0.0,0.5}{#1}}
\newcommand{\mygreen}[1]{\textcolor[rgb]{0.0,0.5,0.0}{#1}}
\newcommand{\mygreenbf}[1]{\textcolor[rgb]{0.0,0.5,0.0}{\bf #1}}
\newcommand{\mypurple}[1]{\textcolor[rgb]{0.5,0.0,0.5}{#1}}
\newcommand{\mypurplebf}[1]{\textcolor[rgb]{0.5,0.0,0.5}{\bf #1}}

\begin{document}
%%%%%%%%%%%%%%%%%%%%%%%%%%%%%%%%%%%%%%%%%%%%%%%%%%%%%%%%%%%%%%%%%%%%%%%%%%%
% ここから課題レポートの記述
%%%%%%%%%%%%%%%%%%%%%%%%%%%%%%%%%%%%%%%%%%%%%%%%%%%%%%%%%%%%%%%%%%%%%%%%%%%

\begin{center} 
{\large \bf 知的プログラミング演習I 第8回レポート}
\end{center} %

\begin{flushright} 
2025年7月27日 % Date
\hskip 1mm
学籍番号 35714121% 学籍番号
\hskip 1mm
氏名福富隆大 % 氏名
\end{flushright} % Name

%%%%%%%%%%%%%%%%%%%%%%%%%%%%%%%%%%%%%%%%%%%%%%%%%%%%%%%%%%%%%%%%%%%%%%%%%%%
\section{レポートのテーマ}
%%%%%%%%%%%%%%%%%%%%%%%%%%%%%%%%%%%%%%%%%%%%%%%%%%%%%%%%%%%%%%%%%%%%%%%%%%%

本レポートでは、課題選択肢の中から「生成AI (Generative AI)」をテーマとして選択する。ChatGPTやStable Diffusionなどの生成系AIについて調査し、その仕組みや社会への影響について考察する。

%%%%%%%%%%%%%%%%%%%%%%%%%%%%%%%%%%%%%%%%%%%%%%%%%%%%%%%%%%%%%%%%%%%%%%%%%%%
\section{調査した深層学習モデルとその原理}
%%%%%%%%%%%%%%%%%%%%%%%%%%%%%%%%%%%%%%%%%%%%%%%%%%%%%%%%%%%%%%%%%%%%%%%%%%%

\subsection{ChatGPT}
ChatGPTは、OpenAIが開発した会話型AIである。大量のテキストデータを学習し、人間のような自然な文章を生成できる。基本的な仕組みは、文章の中で次に来る単語を予測することを繰り返すことで文章を作り出している。

\subsection{Stable Diffusion}
Stable Diffusionは、文章から画像を生成するAIツールである。ランダムなノイズから始めて、徐々にノイズを減らしながら目的の画像を作り出す。テキストの内容に合わせて画像を調整する機能も持っている。

%%%%%%%%%%%%%%%%%%%%%%%%%%%%%%%%%%%%%%%%%%%%%%%%%%%%%%%%%%%%%%%%%%%%%%%%%%%
\section{実応用上の利点}
%%%%%%%%%%%%%%%%%%%%%%%%%%%%%%%%%%%%%%%%%%%%%%%%%%%%%%%%%%%%%%%%%%%%%%%%%%%

生成AIの主な利点は以下の通りである:
\begin{itemize}
\item \textbf{創作活動の支援}: イラストや文章の制作を手助けし、アイデア出しにも活用できる
\item \textbf{業務効率化}: レポート作成や資料作成の時間短縮が可能
\item \textbf{教育支援}: 学習の補助ツールとして活用できる
\item \textbf{アクセシビリティ}: 専門知識がなくても高品質なコンテンツを作成可能
\end{itemize}

%%%%%%%%%%%%%%%%%%%%%%%%%%%%%%%%%%%%%%%%%%%%%%%%%%%%%%%%%%%%%%%%%%%%%%%%%%%
\section{問題点と課題}
%%%%%%%%%%%%%%%%%%%%%%%%%%%%%%%%%%%%%%%%%%%%%%%%%%%%%%%%%%%%%%%%%%%%%%%%%%%

一方で、生成AIには以下の問題もある:
\begin{itemize}
\item \textbf{偽情報の拡散}: 間違った情報を事実のように生成する可能性
\item \textbf{著作権の問題}: 既存の作品に似た内容を無断で生成する恐れ
\item \textbf{雇用への影響}: クリエイターやライターの仕事が減る可能性
\item \textbf{悪用のリスク}: フェイク画像やなりすましに使われる危険性
\end{itemize}

実際に、一部の学校ではAIの使用を制限したり、SNSでAI生成画像による詐欺が発生したりしている。

%%%%%%%%%%%%%%%%%%%%%%%%%%%%%%%%%%%%%%%%%%%%%%%%%%%%%%%%%%%%%%%%%%%%%%%%%%%
\section{考察および感想}
%%%%%%%%%%%%%%%%%%%%%%%%%%%%%%%%%%%%%%%%%%%%%%%%%%%%%%%%%%%%%%%%%%%%%%%%%%%

生成AIは便利で革新的な技術だが、使い方を間違えると社会に悪影響を与える可能性がある。特に学生の立場から考えると、AIに頼りすぎて自分で考える力が衰えないよう注意が必要だと感じた。

また、クリエイターの権利を守りながらAIの恩恵を受ける方法を見つけることが重要だと思う。技術の進歩と社会のルール作りがバランス良く進むことが、AI時代を豊かに生きるために必要だろう。

今後は、AIを適切に活用しながら人間の創造性を伸ばす教育や、AIと人間が協力する新しい働き方が求められると考える。

%%%%%%%%%%%%%%%%%%%%%%%%%%%%%%%%%%%%%%%%%%%%%%%%%%%%%%%%%%%%%%%%%%%%%%%%%%%
\section{参考文献}
%%%%%%%%%%%%%%%%%%%%%%%%%%%%%%%%%%%%%%%%%%%%%%%%%%%%%%%%%%%%%%%%%%%%%%%%%%%

\begin{enumerate}
\item ChatGPT(チャットGPT)とは?使い方や始め方、日本語対応アプリでできることも紹介! \url{https://aismiley.co.jp/ai_news/chatgpt-tsukattemita/}
\item 初等中等教育段階における生成AIに関するこれまでの取組み \url{https://www.mext.go.jp/content/20240725-mxt_jogai01-000037149_21.pdf}
\item 図で見てわかる!画像生成AI「Stable Diffusion」の仕組み \url{https://qiita.com/ps010/items/ea4e8ddeff4de62d1ab1}
\item AI生成画像の見分け方、注目ポイントを専門家が伝授 \url{https://www.nikkei.com/article/DGXZQOUC12AQW0S3A510C2000000/}
\item 生成AIの最新トレンドと企業活用の実践ガイド \url{https://usknet.com/dxgo/contents/dx-technology/the-latest-trends-in-generative-ai-and-practical-guide-for-business/}
\end{enumerate}

%%%%%%%%%%%%%%%%%%%%%%%%%%%%%%%%%%%%%%%%%%%%%%%%%%%%%%%%%%%%%%%%%%%%%%%%%%%
% ここまで課題レポートの記述
%%%%%%%%%%%%%%%%%%%%%%%%%%%%%%%%%%%%%%%%%%%%%%%%%%%%%%%%%%%%%%%%%%%%%%%%%%%
\end{document}
