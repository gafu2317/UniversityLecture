\documentclass{jarticle}

%%%%%%%%%%%%%%%%%%%%%%%%%%%%%%%%%%%%%%%%%%%%%%%%%%%%%%%%%%%%%%%%%%%%%%%%%%%
% packages
%%%%%%%%%%%%%%%%%%%%%%%%%%%%%%%%%%%%%%%%%%%%%%%%%%%%%%%%%%%%%%%%%%%%%%%%%%%
\usepackage{amsmath}
\usepackage{amssymb}
\usepackage{amsfonts}
\usepackage[dvipdfmx]{graphicx}
\usepackage[dvipdfmx]{color}
\usepackage{graphicx}
\usepackage{bm}
\usepackage{url}

%%%%%%%%%%%%%%%%%%%%%%%%%%%%%%%%%%%%%%%%%%%%%%%%%%%%%%%%%%%%%%%%%%%%%%%%%%%
% format stuffs
%%%%%%%%%%%%%%%%%%%%%%%%%%%%%%%%%%%%%%%%%%%%%%%%%%%%%%%%%%%%%%%%%%%%%%%%%%%
\setlength{\oddsidemargin}{0.455cm} 
\setlength{\evensidemargin}{0.455cm} 
\setlength{\textwidth}{15.5cm} 
\setlength{\textheight}{22.54cm}
\setlength{\headheight}{0mm}
\setlength{\headsep}{0mm}
\setlength{\topskip}{0mm}
\setcounter{topnumber}{100}
\setcounter{bottomnumber}{100}
\setcounter{totalnumber}{100}
\renewcommand{\topfraction}{1.0}
\renewcommand{\bottomfraction}{1.0}
\renewcommand{\textfraction}{0.0}
\renewcommand{\floatpagefraction}{0.0}
\renewcommand{\baselinestretch}{1.0}
\pagestyle{empty}

%%%%%%%%%%%%%%%%%%%%%%%%%%%%%%%%%%%%%%%%%%%%%%%%%%%%%%%%%%%%%%%%%%%%%%%%%%%
% math symbols and commands
%%%%%%%%%%%%%%%%%%%%%%%%%%%%%%%%%%%%%%%%%%%%%%%%%%%%%%%%%%%%%%%%%%%%%%%%%%%
\newcommand{\eq}[1]{(\ref{#1})}
\newcommand{\mtx}[2]{\left[\begin{array}{#1} #2 \end{array}\right]}
\newcommand{\mycase}[1]{\left\{\begin{array}{ll} #1 \end{array} \right.}
\newcommand{\mb}[1]{\mbox{\boldmath$#1$}}
\newcommand{\lw}[1]{\smash{\lower2.ex\hbox{#1}}}
\newcommand{\zero}{\mathbf{0}}
\newcommand{\one}{\mathbf{1}}
\newcommand{\eps}{\varepsilon}

%%%%%%%%%%%%%%%%%%%%%%%%%%%%%%%%%%%%%%%%%%%%%%%%%%%%%%%%%%%%%%%%%%%%%%%%%%%
% colors
%%%%%%%%%%%%%%%%%%%%%%%%%%%%%%%%%%%%%%%%%%%%%%%%%%%%%%%%%%%%%%%%%%%%%%%%%%%
\newcommand{\myred}[1]{\textcolor{red}{#1}}
\newcommand{\myredbf}[1]{\textcolor{red}{\bf #1}}
\newcommand{\myblue}[1]{\textcolor{blue}{#1}}
\newcommand{\mybluebf}[1]{\textcolor{blue}{\bf #1}}
\newcommand{\mydarkblue}[1]{\textcolor[rgb]{0.0,0.0,0.5}{#1}}
\newcommand{\mygreen}[1]{\textcolor[rgb]{0.0,0.5,0.0}{#1}}
\newcommand{\mygreenbf}[1]{\textcolor[rgb]{0.0,0.5,0.0}{\bf #1}}
\newcommand{\mypurple}[1]{\textcolor[rgb]{0.5,0.0,0.5}{#1}}
\newcommand{\mypurplebf}[1]{\textcolor[rgb]{0.5,0.0,0.5}{\bf #1}}

\begin{document}
%%%%%%%%%%%%%%%%%%%%%%%%%%%%%%%%%%%%%%%%%%%%%%%%%%%%%%%%%%%%%%%%%%%%%%%%%%%
% ここから課題レポートの記述
%%%%%%%%%%%%%%%%%%%%%%%%%%%%%%%%%%%%%%%%%%%%%%%%%%%%%%%%%%%%%%%%%%%%%%%%%%%

\begin{center} 
{\large \bf 知的プログラミング演習I 第8回レポート}
\end{center} %

\begin{flushright} 
2025年7月27日 % Date
\hskip 1mm
学籍番号 35714121% 学籍番号
\hskip 1mm
氏名福富隆大 % 氏名
\end{flushright} % Name

%%%%%%%%%%%%%%%%%%%%%%%%%%%%%%%%%%%%%%%%%%%%%%%%%%%%%%%%%%%%%%%%%%%%%%%%%%%
\section{レポートのテーマ}
%%%%%%%%%%%%%%%%%%%%%%%%%%%%%%%%%%%%%%%%%%%%%%%%%%%%%%%%%%%%%%%%%%%%%%%%%%%

本レポートでは、課題選択肢の中から「多様な分野への広がり(生成AI以外)」をテーマとして選択する。遺伝子データ解析、音声認識、自然言語処理などの分野における深層学習の最近の実応用について調査し、その利点や問題点について考察する。

%%%%%%%%%%%%%%%%%%%%%%%%%%%%%%%%%%%%%%%%%%%%%%%%%%%%%%%%%%%%%%%%%%%%%%%%%%%
\section{調査した深層学習モデルとその原理}
%%%%%%%%%%%%%%%%%%%%%%%%%%%%%%%%%%%%%%%%%%%%%%%%%%%%%%%%%%%%%%%%%%%%%%%%%%%

\subsection{遺伝子データ解析におけるCNN}
遺伝子データ解析では、畳み込みニューラルネットワーク(CNN)が活用されている。遺伝子配列をデジタルデータに変換し、配列の中から病気に関連する特徴的なパターンを発見する。これにより、がんのリスク予測や薬の効果予測などが可能になっている。

\subsection{音声認識のRNN/LSTM}
音声認識技術では、循環ニューラルネットワーク(RNN)や長短期記憶(LSTM)が使われている。音声の時系列データを処理し、話された言葉を文字に変換する。スマートフォンの音声アシスタントや自動翻訳などで実用化されている。

\subsection{自然言語処理のTransformer}
自然言語処理では、Transformerアーキテクチャが革新をもたらした。文章の構造や意味を理解し、翻訳や要約、質問応答などを高精度で実行できる。Google翻訳やDeepLなどのサービスで使われている。

%%%%%%%%%%%%%%%%%%%%%%%%%%%%%%%%%%%%%%%%%%%%%%%%%%%%%%%%%%%%%%%%%%%%%%%%%%%
\section{実応用での利点}
%%%%%%%%%%%%%%%%%%%%%%%%%%%%%%%%%%%%%%%%%%%%%%%%%%%%%%%%%%%%%%%%%%%%%%%%%%%

これらの技術の主な利点は以下の通りである:

\begin{itemize}
\item \textbf{医療分野}: 病気の早期発見や個別化医療の実現
\item \textbf{コミュニケーション}: 言語の壁を越えた国際的な交流の促進
\item \textbf{アクセシビリティ}: 聴覚障害者や視覚障害者への支援技術の向上
\item \textbf{研究の加速}: 大量のデータを高速で処理し、新しい発見を促進
\end{itemize}

%%%%%%%%%%%%%%%%%%%%%%%%%%%%%%%%%%%%%%%%%%%%%%%%%%%%%%%%%%%%%%%%%%%%%%%%%%%
\section{欠点と問題点}
%%%%%%%%%%%%%%%%%%%%%%%%%%%%%%%%%%%%%%%%%%%%%%%%%%%%%%%%%%%%%%%%%%%%%%%%%%%

一方で、以下の課題も存在する:

\begin{itemize}
\item \textbf{データの偏り}: 学習データに偏りがあると、特定の集団に不利な結果を生む
\item \textbf{プライバシーの問題}: 医療データや音声データの取り扱いに注意が必要
\item \textbf{誤診のリスク}: AIの判断を過信すると、重大な見落としが発生する可能性
\item \textbf{技術格差}: 高度な技術を持つ国や組織とそうでない場所での格差拡大
\end{itemize}

実際に、医療AIで特定の人種に対する診断精度が低かったり、音声認識で方言や女性の声の認識率が低いといった問題が報告されている。

%%%%%%%%%%%%%%%%%%%%%%%%%%%%%%%%%%%%%%%%%%%%%%%%%%%%%%%%%%%%%%%%%%%%%%%%%%%
\section{考察}
%%%%%%%%%%%%%%%%%%%%%%%%%%%%%%%%%%%%%%%%%%%%%%%%%%%%%%%%%%%%%%%%%%%%%%%%%%%

深層学習の多様な分野への応用は、社会に大きな恩恵をもたらしている。特に医療分野では、これまで人間では発見困難だった病気のパターンを見つけることができるようになり、多くの人の命を救う可能性がある。

しかし、技術の恩恵を受けられる人とそうでない人の格差が生まれる危険性もある。AIシステムの公平性を保つためには、多様なデータでの学習と継続的な改善が必要である。

また、AIに頼りすぎず、人間の専門知識と組み合わせて使うことが重要だと感じた。技術は道具であり、最終的な判断は人間が行うべきだろう。

今後は、より多くの人が技術の恩恵を受けられるよう、技術の民主化と教育の充実が求められると考える。

%%%%%%%%%%%%%%%%%%%%%%%%%%%%%%%%%%%%%%%%%%%%%%%%%%%%%%%%%%%%%%%%%%%%%%%%%%%
\section{参考文献}
%%%%%%%%%%%%%%%%%%%%%%%%%%%%%%%%%%%%%%%%%%%%%%%%%%%%%%%%%%%%%%%%%%%%%%%%%%%

\begin{enumerate}
\item 医療AIの現状と今後の展望 \url{https://www.science.co.jp/annotation_blog/41025/}
\item AI技術を活用した音声認識とは?仕組みや活用例、今後の課題まで \url{https://gijiroku.ai/blog/artificial-intelligence/2535}
\item 深層学習による空間的な遺伝子発現量の予測に成功 \url{https://www.ims.u-tokyo.ac.jp/imsut/jp/about/press/page_00159.html}
\item 自然言語処理(NLP)とは?最新の市場動向・仕組み・技術を解説 \url{https://syp.vn/jp/article/what-is-NPL}
\item 医療現場に革命!生成AIが変える診断と治療 \url{https://aka-link.net/generation-ai-medical-care/}
\end{enumerate}

%%%%%%%%%%%%%%%%%%%%%%%%%%%%%%%%%%%%%%%%%%%%%%%%%%%%%%%%%%%%%%%%%%%%%%%%%%%
% ここまで課題レポートの記述
%%%%%%%%%%%%%%%%%%%%%%%%%%%%%%%%%%%%%%%%%%%%%%%%%%%%%%%%%%%%%%%%%%%%%%%%%%%
\end{document}