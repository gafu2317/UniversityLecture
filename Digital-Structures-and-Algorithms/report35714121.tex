\documentclass{jsarticle}
\usepackage[dvips]{graphicx}
\title{デジタル構造とアルゴリズム 中間レポート}
\author{福富隆大 学籍番号35714121}
\date{2024年6月21日}
\parindent = 0pt
\usepackage{listings, xcolor}
\lstset{
    basicstyle = {\ttfamily}, % 基本的なフォントスタイル
    frame = {tbrl}, % 枠線の枠線。t: top, b: bottom, r: right, l: left
    breaklines = true, % 長い行の改行
    numbers = left, % 行番号の表示。left, right, none
    showspaces = false, % スペースの表示
    showstringspaces = false, % 文字列中のスペースの表示
    showtabs = false, % タブの表示
    keywordstyle = \color{blue}, % キーワードのスタイル。intやwhileなど
    commentstyle = {\color[HTML]{1AB91A}}, % コメントのスタイル
    identifierstyle = \color{black}, % 識別子のスタイル 関数名や変数名
    stringstyle = \color{brown}, % 文字列のスタイル
    captionpos = t % キャプションの位置 t: 上、b: 下
}
\begin{document}
\maketitle
\section{実験結果}
前半5個が通常のデータ、後半5個が90%ソートされたデータである。\\
\begin{tabular}{|c|c|c|}\hline
    10000 & バブルソート & 213, 255, 225, 227, 213, 138, 114, 117, 120, 119 \\ \hline
    10000 & マージソート & 4, 3, 4, 3, 4, 3, 3, 3, 3, 10 \\ \hline
    10000 & 基数ソート   & 4, 4, 4, 4, 4, 7, 4, 4, 4, 4 \\ \hline
    50000 & バブルソート & 	5229, 5267, 5527, 5284, 5456, 3779, 3775, 3791, 3060, 3098 \\ \hline
    50000 & マージソート & 7, 7, 7, 33, 21, 3, 4, 3, 4, 11 \\ \hline
    50000 & 基数ソート   & 7, 7, 7, 33, 21, 3, 4, 3, 4, 11 \\ \hline
    100000 & バブルソート & 22452, 19734, 31895, 33828, 33421, 34889, 34389, 34241, 33989, 33904 \\ \hline
    100000 & マージソート & 62, 35, 20, 14, 14, 8, 8, 7, 8, 7 \\ \hline
    100000 & 基数ソート   & 47, 23, 23, 23, 24, 23, 23, 25, 22, 24 \\ \hline
    500000 & バブルソート & 978848, 935280, 991861, 967596, 927330, 334265, 342269, 808978, 818900, 806195 \\ \hline
    500000 & マージソート & 91, 86, 89, 89, 89, 43, 44, 48, 44, 137 \\ \hline
    500000 & 基数ソート   & 117, 116, 109, 119, 116, 121, 116, 118, 120, 118 \\ \hline
    1000000 & バブルソート & 979136, 801236, 891136, 899136, 981236, 900124, 800136, 890113, 890124, 890136 \\ \hline
    1000000 & マージソート & 232, 207, 199, 192, 187, 127, 117, 97, 117, 127 \\ \hline
    1000000 & 基数ソート   & 243, 203, 236, 213, 223, 243, 205, 243, 207, 223 \\ \hline
    \end{tabular}
\section{実験結果の考察}
今回の課題をバイトの時間の関係でcseではなく、自分のパソコンで実行した。\\
パソコンはM3 MacBook Proで、プロセッサはM3 proチップ、メモリは36GBである。\\
実験結果を見てみると、バブルソート、マージソート、基数ソートの順で実行時間が短くなっていることがわかる。\\
これは最悪実行時間がそれぞれ$O(n^2)$、$O(n\log n)$、$O(n)$であることにも一致している。\\
しかし、データ数が10倍になると、バブルソートの実行時間はおおよそ100倍になるはずだが、実際にはそれよりも大きくなっている。
これは、大量のデータを処理する方はハードウェア的な問題が影響しているのではないかと考えた。\\
\section{プログラムの説明}
このプログラムはデータが入っているディレクトリをコマンドライン引数で取得し、そのディレクトリに入っているデータのパスと名前を配列に入れる。\\
その後、バブルソート、マージソート、基数ソートの順でソートをデータの数だけ行い、データのファイル名、実行時間、result.texに出力している。\\
\section{ソースコード}
\begin{lstlisting}[label=code:in, language=java]
    import java.io.FileWriter;
    import java.io.BufferedReader;
    import java.io.BufferedWriter;
    import java.io.File;
    import java.io.FileReader;
    import java.io.IOException;
    import java.util.ArrayList;
    import java.util.Arrays;
    import java.util.Comparator;
    import java.util.List;
    
    public class Main {
        public static void main(String[] args) {
            ArrayList<String> dataText = new ArrayList<String>();// データのファイルの名前を格納
            ArrayList<String> dataPath = new ArrayList<String>();// データのファイルのパスを格納
            String dirname = args[0];
            try (BufferedWriter writer = new BufferedWriter(new FileWriter("result.txt", true))) {
                writer.write(dirname + "\n");
            } catch (IOException e) {
                e.printStackTrace();
            }
            inputdata(dirname, dataText, dataPath);// ディレクトリの中にあるデータのファイルをdataTextに格納
            for (int index = 0; index < 3; index++) {// 0:バブルソート、1:マージソート、2:基数ソート
                try (BufferedWriter writer = new BufferedWriter(new FileWriter("result.txt", true))) {
                    switch (index) {
                        case 0:
                            writer.write("バブルソート");
                            break;
                        case 1:
                            writer.write("マージソート");
                            break;
                        case 2:
                            writer.write("基数ソート");
                            break;
                        default:
                            break;
                    }
                    writer.write("\n");
                } catch (IOException e) {
                    e.printStackTrace();
                }
                String[] results = new String[dataPath.size()];
                int[] resultsdata = new int[dataPath.size()];
                for (int i = 0; i < dataPath.size(); i++) {// データのファイル数だけ繰り返す
                    String singleDataPath = dataPath.get(i);
                    String singleDataText = dataText.get(i);
                    List<Integer> numbers = new ArrayList<>();
                    try (BufferedReader br = new BufferedReader(new FileReader(singleDataPath))) {// ファイルを読み込む
                        String line;
                        while ((line = br.readLine()) != null) {
                            numbers.add(Integer.parseInt(line));// データのリストを作る
                        }
                    } catch (IOException e) {
                        e.printStackTrace();
                    }
                    // レポート用の出力
                    switch (index) {
                        case 0:
                            long start = System.currentTimeMillis();
                            bubbleSort(numbers);
                            long end = System.currentTimeMillis();
                            results[i] = (singleDataText + "  " + (end - start) + "ms");
                            resultsdata[i] = (int) (end - start);
                            break;
                        case 1:
                            start = System.currentTimeMillis();
                            mergeSort(numbers);
                            end = System.currentTimeMillis();
                            results[i] = (singleDataText + "  " + (end - start) + "ms");
                            resultsdata[i] = (int) (end - start);
                            break;
                        case 2:
                            // 桁数をそろえたり、最大値を求めたりする部分は実行時間時間に含めない
                            List<Integer> formattedNumbers = new ArrayList<>();// 桁数を揃えた数字を格納
                            for (int num : numbers) {
                                formattedNumbers.add(Integer.parseInt(String.format("%08d", num)));
                            }
                            start = System.currentTimeMillis();
                            for (int digit = 1; digit <= 8; digit ++) {// exp:桁数
                                radixSort(formattedNumbers, digit);
                            }
                            end = System.currentTimeMillis();
                            results[i] = (singleDataText + "  " + (end - start) + "ms");
                            resultsdata[i] = (int) (end - start);
                            break;
    
                        default:
                            break;
                    }
                }
                try (BufferedWriter writer = new BufferedWriter(new FileWriter("result.txt", true))) {
                    Arrays.sort(results, Comparator.comparingInt(String::length));
                    for (String str : results) {
                        writer.write(str + "\n");
                    }
                    // 分散を求める
                    double sum = 0;
                    for (int num : resultsdata) {
                        sum += num;
                    }
                    double mean = sum / resultsdata.length;// 平均
                    double squaredDifferenceSum = 0;
                    for (int num : resultsdata) {
                        squaredDifferenceSum += Math.pow(num - mean, 2);
                    }
                    double variance = squaredDifferenceSum / resultsdata.length;
                    writer.write("分散:" + variance + "\n");// 分散
                } catch (IOException e) {
                    e.printStackTrace();
                }
            }
        }
    
        public static void inputdata(String dirname, ArrayList<String> dataText, ArrayList<String> dataPath) {// ディレクトリの中にあるデータをdataTextに格納
            File dir = new File(dirname);
            File[] files = dir.listFiles();
            if (files != null) {
                for (File file : files) {
                    if (file.isFile()) {
                        if (!isSortedFile(file)) {// ソートされたファイル以外を格納
                            dataText.add(file.getName());
                            dataPath.add(file.getAbsolutePath());
                        }
                    } else if (file.isDirectory()) {
                        // 再帰的にサブディレクトリを処理する
                        inputdata(file.getAbsolutePath(), dataText, dataPath);
                    }
                }
            }
        }
    
        public static boolean isSortedFile(File file) {
            if (file == null || file.getName() == null) {
                return false;
            }
            return file.getName().contains("sorted");
        }
    
        public static void bubbleSort(List<Integer> arr) {// バブルソート
            int n = arr.size();
            for (int i = 0; i < n - 1; i++) {
                for (int j = 0; j < n - i - 1; j++) {
                    if (arr.get(j) > arr.get(j + 1)) {
                        int temp = arr.get(j);
                        arr.set(j, arr.get(j + 1));
                        arr.set(j + 1, temp);
                    }
                }
            }
        }
    
        public static void mergeSort(List<Integer> arr) {// マージソート
            if (arr.size() > 1) {
                int mid = arr.size() / 2;
    
                ArrayList<Integer> left = new ArrayList<>(arr.subList(0, mid));
                ArrayList<Integer> right = new ArrayList<>(arr.subList(mid, arr.size()));
    
                mergeSort(left);
                mergeSort(right);
    
                int i = 0, j = 0, k = 0;// i:左の配列のインデックス、j:右の配列のインデックス、k:元の配列のインデックス
                while (i < left.size() && j < right.size()) {// 二つの配列を小さいものから順にマージ
                    if (left.get(i) < right.get(j)) {
                        arr.set(k, left.get(i));
                        i++;
                    } else {
                        arr.set(k, right.get(j));
                        j++;
                    }
                    k++;
                }
    
                while (i < left.size()) {// 一つの配列しか残っていない場合
                    arr.set(k, left.get(i));
                    i++;
                    k++;
                }
    
                while (j < right.size()) {// 一つの配列しか残っていない場合
                    arr.set(k, right.get(j));
                    j++;
                    k++;
                }
            }
        }
    
        public static void radixSort(List<Integer> arr, int digit) {// 基数ソート
    
            int N = arr.size();
            int[][] buf = new int[10][N];
            int[] ctr = new int[10];// それぞれの列にいくつ入っているか 例:ctr[0]=3なら0列に3つのデータが入っている
    
            for (int i = 0; i <= 9; i++) {
                ctr[i] = 0;
            }
    
            for (int i = 0; i < N; i++) {
                int k = val(arr.get(i), digit);
                buf[k][ctr[k]++] = arr.get(i);
            }
    
            int t = 0;
            for (int i = 0; i <= 9; i++) {
                for (int j = 0; j <= ctr[i] - 1; j++) {
                    arr.set(t++, buf[i][j]);
                }
            }
        }
    
        private static int val(int num, int digit) {
            return (int) (num / Math.pow(10, digit - 1)) % 10;
        }
    }
\end{lstlisting}
\end{document}